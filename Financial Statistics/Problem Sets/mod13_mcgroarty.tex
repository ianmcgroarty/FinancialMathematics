\documentclass[12pt]{article}
\usepackage[T1]{fontenc}
%\usepackage[latin9]{inputenc}
\usepackage[utf8]{inputenc}
\usepackage[english]{babel}
\usepackage{amsmath}
\usepackage{amsfonts}
\usepackage{amssymb}
\usepackage{setspace}
\usepackage{rotating}
\usepackage{graphics}
\usepackage[round]{natbib}
%\usepackage{graphicx}
%\usepackage{float} 				%allows you to float images
\usepackage{latexsym}
\usepackage{bbding}
%\usepackage {moresize}
\usepackage{listings}
\usepackage{bbding}
\usepackage{blindtext}
\usepackage{hhline}
%\usepackage{tikz}
%\usetikzlibrary{shapes,backgrounds}
%\usepackage{pgfplots}
%\usetikzlibrary{arrows}
\usepackage{enumitem}
\doublespacing
%\usepackage{geometry}
\usepackage{amsthm}
\usepackage{color}
%\usepackage{array,multirow}
\usepackage{caption}
%\usepackage{subcaption}
%\usepackage{pst-plot}
%	\psset{xunit=15mm}
%\geometry{verbose,tmargin=1in,bmargin=1in,lmargin=1in,rmargin=1in}
\setlength{\parskip}{\bigskipamount}
\setlength{\parindent}{0pt}

\newenvironment{problem}[2][Problem]{\begin{trivlist}
\item[\hskip \labelsep {\bfseries #1}\hskip \labelsep {\bfseries #2.}]}{\end{trivlist}}

\title{Problem Set 13 \thanks{Problem list 13.3.2, 13.3.6, 14.2.6 }}
\author{Ian McGroarty \\
	Course Number: 625.603}
\date{May 9, 2019}

\begin{document}

\maketitle
\newpage

\begin{problem}{13.3.2} From Table 13.3.2:
$$ \Sigma_{i=1}^{13} d_i = -42 \text{ \ \ \     and  \ \ \   } \Sigma_{i=1}^{13} d_i^2 = 216 $$
Therefore: 
$$ \bar{d} = \frac{1}{13}(-42) = -3.23. \text{ \ \ \  and \ \ \ } s_d^2 = \frac{13(216)-(-42)^2}{13(12)}=6.69 $$
This produces a t ratio of:
$$ t = \frac{ -3.23 }{\sqrt{6.69}/\sqrt{13}} = -4.502$$
\end{problem}
\begin{table}[h!]
\centering
\begin{tabular}{lccr}
x  &	y   &	x-y &	$d^2$ \\
\hline
2  &	3  &	-1 &	1 \\
3  &	11  &	-8 &	64 \\
5  &	10  &	-5 &	25 \\
3  &	5   &	-2 &	4 \\
2  &	5   &	-3 &	9 \\
1  &	4  &	-3 &	9 \\
1  &	2  &	-1 &	1 \\
5  & 	7  &	-2 &	4 \\ 
3  &	 5  &	-2 &	4 \\
1  &	4  &	-3 &	9 \\
7  &	8  &	-1 &	1 \\
3  &	12 &	-9 &	81 \\ 
5  &	7 &	-2 &	4 \\
\hline
&  &		-42 &	216
\end{tabular}
\caption*	{Table 13.3.2}
\end{table}
With 13 degrees of freedom. $\pm t_{0.025,13} = \pm 2.16$. By Theorem 13.3.1 (pg 629) we can reject the null hypothesis and conclude the difference between the number of trials needed to learn depth perception for unmothered versus mothered lambs is statistically significant. 

%%%%%%%%%%%%%%%%%%%%%%%%%%%%%%%%%%%%%%%%%%%%%%
%%%%%%%%%%%%%%%%%%%%%%%%%%%%%%%%%%%%%%%%%%%%%%
%%%%%%%%%%%%%%%%%%%%%%%%%%%%%%%%%%%%%%%%%%%%%%

\begin{problem}{13.3.6} In Case Study 13.3.1 we saw that $\bar{d} =0.47$ and $ S_D^2 = 0.662$ and $t_{0.025,9} = \pm 2.2622$ We can use this information to find the 95\% confidence interval: 
\begin{align*}
(\bar{d} - t_{\alpha/2 , b-1} \cdot \frac{S_D}{\sqrt{b}} &, \bar{d} - t_{\alpha/2 , b-1} \cdot \frac{S_D}{\sqrt{b}}) \\
(0.47 - 2.2622 \cdot \frac{0.813}{\sqrt{10}} &, 0.47 + 2.2622 \cdot \frac{0.813}{\sqrt{10}}) \\
(-0.10433 &, 1.044331) 
\end{align*}
\end{problem}


%%%%%%%%%%%%%%%%%%%%%%%%%%%%%%%%%%%%%%%%%%%%%%
%%%%%%%%%%%%%%%%%%%%%%%%%%%%%%%%%%%%%%%%%%%%%%
%%%%%%%%%%%%%%%%%%%%%%%%%%%%%%%%%%%%%%%%%%%%%%
\newpage
\begin{problem}{14.2.6} Width to length ratios of the of Shoshoni Rectangles:

\begin{table}[h!]
\centering
\begin{tabular}{lccr}
\hline
0.693*  &   0.749*  &	0.654*  & 	0.67*  \\
0.662*  &   0.672*  &	0.615  & 	0.606 \\
0.69*    &     0.628* &	0.668*  & 	0.611 \\
0.606    &   0.609  &	0.601  & 	0.553 \\
0.57      &    0.844*  &	0.576  & 	0.933* \\
\hline
\end{tabular}
\end{table}

Let $\mu $ denote the median length to width ratio. Testing 
\begin{align*}
H_0: & \bar{\mu } = 0.618 \\
H_1: & \bar{\mu } \neq 0.618
\end{align*}
then becomes a way of quantifying the asthetic of rectangles... By inspection, a total of k=11 of the n=20 ratios exceed 0.618. Let $\alpha = 0.05$, with 19 degrees of freedom the critical values are $t_{0.025,19}= \pm 2.0930 $
\begin{align*}
z &= \frac{k-n/2}{\sqrt{n/4}} && \text{Theorem 14.2.1 (pg 640)} \\
&= \frac{11-(20/2)}{\sqrt{20/5}} \\
z &= 0.25 
\end{align*}
Clearly we can not reject the null hypothesis, thus it appears that the Shoshoni Indians also embraced the golden rectanlge as an aesthetic standard. 

\end{problem}


%%%%%%%%%%%%%%%%%%%%%%%%%%%%%%%%%%%%%%%%%%%%%%
%%%%%%%%%%%%%%%%%%%%%%%%%%%%%%%%%%%%%%%%%%%%%%
%%%%%%%%%%%%%%%%%%%%%%%%%%%%%%%%%%%%%%%%%%%%%%
\end{document}


