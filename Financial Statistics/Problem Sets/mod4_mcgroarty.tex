\documentclass[12pt]{article}
\usepackage[T1]{fontenc}
%\usepackage[latin9]{inputenc}
\usepackage[utf8]{inputenc}
\usepackage[english]{babel}
\usepackage{amsmath}
\usepackage{amsfonts}
\usepackage{amssymb}
\usepackage{setspace}
\usepackage{rotating}
\usepackage{graphics}
\usepackage[round]{natbib}
%\usepackage{graphicx}
%\usepackage{float} 				%allows you to float images
\usepackage{latexsym}
\usepackage{bbding}
%\usepackage {moresize}
\usepackage{listings}
\usepackage{bbding}
\usepackage{blindtext}
\usepackage{hhline}
%\usepackage{tikz}
%\usetikzlibrary{shapes,backgrounds}
%\usepackage{pgfplots}
%\usetikzlibrary{arrows}
\usepackage{enumitem}
\doublespacing
%\usepackage{geometry}
\usepackage{amsthm}
\usepackage{color}
%\usepackage{array,multirow}
%\usepackage{subcaption}
%\usepackage{pst-plot}
%	\psset{xunit=15mm}
%\geometry{verbose,tmargin=1in,bmargin=1in,lmargin=1in,rmargin=1in}
\setlength{\parskip}{\bigskipamount}
\setlength{\parindent}{0pt}

\newenvironment{problem}[2][Problem]{\begin{trivlist}
\item[\hskip \labelsep {\bfseries #1}\hskip \labelsep {\bfseries #2.}]}{\end{trivlist}}

\title{Problem Set 4 \thanks{Problem list 3.10.6, 3.10.16, 3.12.6, 3.12.8}}
\author{Ian McGroarty \\
	Course Number: 625.603}
\date{February 28, 2019}

\begin{document}

\maketitle
\newpage
\begin{problem}{3.10.6} Let $Y_1, Y_2, ... , Y_n$ be a random sample from the exponential pdf $f_y(y) = e^{-y}, y \geq 0$. What is the smallest n for which $P(Y_{min} < 0.2) > 0.9$? \\
\textbf{Solution} The evaluated n > 11.513, since x must be an integer (I'm assuming since these are trials) $n\geq 12$.
\begin{align*}
P(Y_{min}<0.2) &= \int_0^{0.2} f_{Y_{min}}(y) \\
&=\int_0^{0.2} n[1-F_Y(y)]^{n-1}f_Y(y) && \text{Theorem 3.10.1.b (pg193)} \\
&=\int_0^{0.2} n[1-(1-e^{-y})]^{n-1}(e^{-y}) \\
&= \int_0^{0.2}n(e^{-y})^{n-1}(e^{-y}) \\
&= \int_0^{0.2}n(e^{-ny}) \\
&= -e^{-ny} \Big|_0^{0.2} \\
&= -e^{-(0.2)n}-(-1) >0.9 \\
&= log(e^{-0.2n}) < log(0.1) \\
n&>\frac{log(0.1)}{-0.2} \approx 11.513
\end{align*}

\end{problem}

\begin{problem}{3.10.16} Suppose a device has three independent components, all of whose lifetimes (in months) are modeled by the exponential pdf, $f_y(y) = e^{-y}, y > 0 $. What is the probability that all three components will fail within two months of one another? \\ \

\textbf{Solution}\footnote{To start, I want to note that understanding of the "memoryless property of the exponential distribution" was critical to even approaching success in this problem. I studied the proof in this pdf, http://www.cs.cmu.edu/afs/cs/academic/class/15750-s19/OldScribeNotes/lecture11.pdf (pg 2). I also used wolframalpha to do some of the calculations that were a to complex for my patience level.} Range = $Y_{max}-Y_{min}=Y^`_3-Y^`_1. \ P(R<r)=0.646$ \\
 The \textit{memoryless property of the exponential distribution}: $$P(X\geq s+t | x \geq s) = P(X\geq t)$$
 This implies that the level of $Y^`_1$ is inconsequential. Thus, we can assume that $Y^`_1 = 0$. In which case we are really only interested in $P(Y^`_{max} < r)$. Thus, we can apply Theorem 3.10.1.a (pg 193) with: n=3, $f_y(y)=e^{-y}$, and $F_Y(y) = \int_0^y f_Y(y)dy = 1-e^{-y}$. 
\begin{align*}
P(Y_{max} < m) &= \int_{-\infty}^m n[F_Y(y)]^{n-1}f_Y(y) \\
P(Y^`_3 < 2) &= \int_0^2  3[1-e^{-y}]^2 e^{-y} &&\text{Enter WolframAlpha} \\
&\approx 0.646. 
\end{align*}
\end{problem}

\begin{problem}{3.12.6} Find $M_Y(t)$ if Y has the pdf: 
$$
f_Y(y) = \left\{
        \begin{array}{ll}
            y, & \quad 0 \leq y \leq 1 \\
            2-y, & \quad 1 \leq y \leq 2 \\
		0,  & \quad \text{elsewhere}
        \end{array}
    \right.
$$

\textbf{Solution} $M_Y(t) = \frac{1-e^t}{t^2}$ \\ Since X is continuous, the second part of Definition 3.12.1 (pg 206) applies: 
\begin{align*}
M_Y(t) &= E(e^{tW})= \int_{-\infty}^{\infty} e^{tw}f_W(w)dw  \\
&= \int_{-\infty}^{0} e^{ty}0 + \int_{0}^{1} e^{ty}y + \int_{1}^{2} e^{ty}(2-y) + \int_{0}^{\infty} e^{ty}0 \\
&= \int_{0}^{1} e^{ty}y + \int_{1}^{2} e^{ty}(2-y)  && \text{I used WolframAlpha here.} \\
&= ( \frac{1}{t}y -\frac{1}{t^2} )e^{ty} \Big|_0^1 + \frac{2}{t} e^{ty} \Big|_1^2 - ( \frac{1}{t} y-\frac{1}{t^2} )e^{ty} \Big|_1^2 \\
&= ( \frac{1}{t} -\frac{1}{t^2} )e^{t} + (\frac{1}{t^2}) + \frac{2}{t} e^{2t} - \frac{2}{t} e^{t} - ( \frac{2}{t} -\frac{1}{t^2} )e^{2t} +( \frac{1}{t} -\frac{1}{t^2} )e^{t} \\
&= \frac{1}{t^2} + \frac{1}{t^2}e^{2t} - \frac{2}{t^2}e^t \\
&= \frac{1}{t^2}(e^t-1)
%&=\frac{e^{tx}(tx-1)}{t^2} \Big|_0^1 + \frac{(y-2)e^{tx}}{t}\Big|_1^2  \\
%&= \frac{1-e^t}{t^2}
\end{align*}
\end{problem}


\begin{problem}{3.12.8} Let Y be a continuous random variable with $f_Y(y) = ye^{-y}, o \leq y$. Show that $M_Y(t) = \frac{1}{(1-t)^2}$ 

\textbf{Solution} Since X is continuous, the second part of Definition 3.12.1 (pg 206) applies:\\
\begin{align*}
M_Y(t) &= E(e^{tW})= \int_{-\infty}^{\infty} e^{tw}f_W(w)dw  \\
&= \int_{0}^{\infty} e^{ty}ye^{-y}  \\
&= \int_{0}^{\infty} e^{ty-y}y \\
&= \frac{e^{(t-1)y}((t-1)y-1)}{(t-1)^2}\Big|_0^{\infty} \\
&= lim_{y\rightarrow \infty} \frac{e^{(t-1)y}((t-1)y-1)}{(t-1)^2}  - lim_{y\rightarrow 0}\frac{e^{(t-1)y}((t-1)y-1)}{(t-1)^2}\\
&= lim_{y\rightarrow \infty} \frac{e^{(t-1)y}((t-1)y-1)}{(t-1)^2}  - \frac{e^{(t-1)0}((t-1)0-1)}{(t-1)^2} \\
&= lim_{y\rightarrow \infty} \frac{e^{U}(U-1)}{(t-1)^2} - \frac{(-1)}{(t-1)^2} && \text{Distribute, let U =(yt-y)} \\
&= \frac{(lim_{y\rightarrow \infty} e^{yt-y})(lim_{y\rightarrow \infty}(yt-y-1))}{(t-1)^2} - \frac{(-1)}{(t-1)^2} && \text{Theorem 20.4 (Ross pg 156)} \\
\end{align*}
\\
It will suffice to show that if $ lim_{y\rightarrow \infty }e^{yt-y} $ converges to 0 then $M_Y(t) \rightarrow \frac{1}{(t-1)^2}$. Since $ 0*(lim_{y\rightarrow \infty}(yt-y-1)) = 0$ and $\frac{0}{(t-1)^2} - \frac{(-1)}{(t-1)^2}=\frac{1}{(t-1)^2}$.
\begin{proof}
Let $\epsilon >0$ and $\delta = \frac{log(\epsilon )}{(t-1)}$. Note that, $ \forall \ t$ such  that  $0<t<1, \ \delta >0$. So if $0<|x|<\delta $ then $|x| < \frac{log(\epsilon ) }{(t-1)} \implies  (t-1)|x| = log(\epsilon )  \implies e^{xt-x}<\epsilon $ Thus, by Theorem 20.6 (Ross pg 159\footnote{Ross, Kenneth: Elementary Analysis the Theory of Calculus. Undergraduate Texts in Mathematics, 2nd edn. Springer, New York/Heidelberg/Berlin (2013)}), $\lim_{x\rightarrow \infty}e^{xt-x} = 0.$
\end{proof}
\end{problem}

\begin{problem}{Assignment} The Inverse Transform Sampling procedure requires $F^-_Y(u)$ where $F^-_Y(u)$ is the inverse of the cumulative distribution function, $F_Y(y)$. We saw in problem 3.10.6 that $F_Y(y) = (1-e^-y)$. Solving for $x=-log(1-u)$ is our inverse CDF function. Now to be honest with you, I couldn't really follow your R code so I wrote my own. Now I'm not really sure what you are looking for to by way of the solution. But I've included my code below, and I am getting very close to 0.9 so I believe it is correct. 
\begin{lstlisting}[language=R]
# inverse transfrom sampling

# This is the number of times you run the process, 
positions <- 1000

# This will be a vector of minimums
samples <- c()

# You add one number to sample each process
# and you want to run the process position times. 
while( length(samples) < positions )
{
#This is the number computed in the problem. 
#The n needed to produce the minimum
num.samples <-  12 

# The uniform distribution from 0 to 1
U           <-  runif(num.samples,0,1)

# The pdf 
Y <- exp(-U)

# The inverse cdf function
X           <- -log(1-U)

# take the minimum
sortedx <- sort(X) 
minx = min(sortedx)
samples <- append(samples,minx)
}

# Get the ratio of samples <0.2 to total samples.
# This is your probability. 
less <- sum(samples <0.2)
ratio <- less/positions


# plot
hist(samples, breaks=30, freq=F, xlab='X', main='Generating Exponential R.V.')
curve(dexp(X, rate=2) , 0, 3, lwd=2, xlab = "", ylab = "", add = T)

\end{lstlisting}

\end{problem}

\end{document}








Integration by parts
\begin{align*}
f(y) = y & \ \ \ df = dy  \\
dg(y) = e^{y(t-1)} & \ \ \ g = \frac{e^{y(t-1)}}{t-1} \\
&= \frac{y(e^{y(t-1)}}{t-1} - \frac{1}{(t-1)}\int e^{y(t-1)} \\
&= \frac{y(e^{yt-y}}{t-1} - \frac{e^{y(t-1)}}{(t-1)^2} \\
 \end{align*}




 This is a matter of using joint probability to determine the pdf of the range: I use equation 3.10.5 (pg 197). With n=3, i=3, j=3, $f_Y(y) = e^{-y}$ and $F_Y(y)=-e^{-y}$.
\begin{align*}
f_{Y_i^`,Y_j^`}(u,v) &= \frac{n!}{(i-1)!(j-i-1)!(n-j)!}[F_Y(u)]^{i-1}[F_Y(v)-F_Y(u)]^{j-i-1} \cdot [1-F_Y(v)]^{n-j}f_Y(u)f_Y(v) \\
&= 6[-e^{-v}+e^{-u}](e^{-v})(e^{-u})
\end{align*}
Find $M_Y(t)$