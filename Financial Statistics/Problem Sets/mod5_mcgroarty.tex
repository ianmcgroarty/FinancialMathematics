\documentclass[12pt]{article}
\usepackage[T1]{fontenc}
%\usepackage[latin9]{inputenc}
\usepackage[utf8]{inputenc}
\usepackage[english]{babel}
\usepackage{amsmath}
\usepackage{amsfonts}
\usepackage{amssymb}
\usepackage{setspace}
\usepackage{rotating}
\usepackage{graphics}
\usepackage[round]{natbib}
%\usepackage{graphicx}
%\usepackage{float} 				%allows you to float images
\usepackage{latexsym}
\usepackage{bbding}
%\usepackage {moresize}
\usepackage{listings}
\usepackage{bbding}
\usepackage{blindtext}
\usepackage{hhline}
%\usepackage{tikz}
%\usetikzlibrary{shapes,backgrounds}
%\usepackage{pgfplots}
%\usetikzlibrary{arrows}
\usepackage{enumitem}
\doublespacing
%\usepackage{geometry}
\usepackage{amsthm}
\usepackage{color}
%\usepackage{array,multirow}
%\usepackage{subcaption}
%\usepackage{pst-plot}
%	\psset{xunit=15mm}
%\geometry{verbose,tmargin=1in,bmargin=1in,lmargin=1in,rmargin=1in}
\setlength{\parskip}{\bigskipamount}
\setlength{\parindent}{0pt}

\newenvironment{problem}[2][Problem]{\begin{trivlist}
\item[\hskip \labelsep {\bfseries #1}\hskip \labelsep {\bfseries #2.}]}{\end{trivlist}}

\title{Problem Set 5 \thanks{Problem list 5.2.12, 5.3.10, 5.3.14, 5.4.20, 5.6.6}}
\author{Ian McGroarty \\
	Course Number: 625.603}
\date{April 4, 2019}

\begin{document}

\maketitle
\newpage
\begin{problem}{5.2.12} A random sample of size $n$ is taken from the pdf: 
\begin{align*}
f_y(y;\theta ) &= \frac{2y}{\theta^2}, \ 0 \leq y \leq \theta \\ 
L(\theta) &= \prod_{i=1}^n f_y(y_i ; \theta) && \text{Def. 5.2.1 (281)} \\ 
&= \theta^{-2n}(2y)^n
\end{align*}
It is easy to see that  $L(\theta)$ is decreasing in $\theta $. To maximize $L(\theta)$ we must minimize $\theta$. Since $y \leq \theta$, $\hat{\theta}  = Y_{max}$.
\end{problem}

\begin{problem}{5.3.10} Babe Ruth batted 0.356 with 192 hits in 540 at-bats. Construct a 95\% confidence interval. \\

\textbf{Solution} To construct a 95\% confidence interval, we can apply Theorem 5.3.1 (299)with k = 192, n = 540, and $z_{\alpha / 2}$ = 1.96.\footnote{From Table A.1 (675)}
\begin{align*}
\frac{k}{n} - z_{\alpha/2}\sqrt{\frac{(k/n)(1-k/n)}{n}}&, \frac{k}{n} + z_{\alpha/2}\sqrt{\frac{(k/n)(1-k/n)}{n}} \\
\frac{192}{540} - 1.96\sqrt{\frac{(192/540)(1-192/540)}{540}}&, \frac{192}{540} + 1.96\sqrt{\frac{(192/540)(1-192/540)}{540}} \\
(0.315&,0.396)
\end{align*}
This is his batting average, so the 95\% confidence interval for hits is $540*(0.315,0.396)$ = (170,214). 
\end{problem}

\begin{problem}{5.3.14} If (0.57,0.63) is a 50\% confindence interval for $p$, find $\frac{k}{n}$ and n. \\ 
\textbf{Solution} The margin of error can be found by using the average distance from the mean, which is the difference in the bounds. $\frac{0.63-0.57}{2} = 0.03 = d = \frac{k}{n}$. To find how many observations were taken we use Definition 5.3.1 (301) $d=\frac{z_{\alpha/2}}{2\sqrt{n}} \implies n = (\frac{z_{\alpha/2}}{2d})^2 = (\frac{0.675}{2*(0.03)})^2 = 126.6 \approx 127$ observations. 
\end{problem}

\begin{problem} {5.4.20} Calculate the relative efficiency (r.e.) of $\hat{\lambda}_1 = X_1$ and $\hat{\lambda}_2=\bar{X_2}$. 
\begin{proof} I mean they are both ``relatively efficient''\footnote{Einstein's Theory of relativity}. \end{proof}
\textbf{Solution} Since it is a Poisson distribution: 
\begin{align*}
 Var_1(X) &= \lambda && \text{Theorem 4.2.4 (224)} \\ 
 Var_2(\bar{X}) &= Var(\frac{X_1}{n} + \cdots + \frac{X_n}{n})  \\ 
 &= \frac{Var(x)}{n} = \frac{\lambda}{n}
 \end{align*}
 The relative efficiency is defined as: 
 \begin{align*}
 r.e. (\hat{\lambda}_1) &= \frac{Var(\hat{\lambda}_2)}{Var(\hat{\lambda}_1)}  && \text{Def. 5.4.2 (314)} \\
 & = \frac{\lambda /n}{\lambda} = \frac{1}{n}
\end{align*}

 
\end{problem}

\begin{problem}{5.6.6} Is $W = \prod_{i=1}^nY_i$ a sufficient statistic for $\theta $. 
\begin{align*}
L(\theta) &= \prod_{i=1}^n f_Y(y;\theta) &\text{Def. 5.6.1 (322)} \\
&= \prod_{i=1}^n \theta y^{\theta - 1} \\
&= \theta^n (\prod_{i=1}^n y_i)^{\theta-1} 
\end{align*}
We can see that $L(\theta)$ depends on $\theta$ but only depends on y through the values of $\prod_{i=1}^ny_i$ . So define $h(Y) = \prod_{i=1}^ny_i$ and $g(h(x);\theta) = \theta^n (\prod_{i=1}^n y_i)^{\theta-1} $ and $b(k_1,\cdots , k_n)=1$. By theorem 5.6.1 (324) $W = \prod_{i=1}^nY_i$ is sufficient for $\theta $
\end{problem}

\end{document}


