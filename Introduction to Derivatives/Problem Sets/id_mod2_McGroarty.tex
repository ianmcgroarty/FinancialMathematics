\documentclass[12pt]{article}
\usepackage[T1]{fontenc}
%\usepackage[latin9]{inputenc}
\usepackage[utf8]{inputenc}
\usepackage[english]{babel}
\usepackage{amsmath}
\usepackage{amsfonts}
\usepackage{amssymb}
%\usepackage{setspace}
\usepackage{rotating}
\usepackage{graphics}
\usepackage{eurosym}
\usepackage[round]{natbib}
%\usepackage{graphicx}
%\usepackage{float} 				%allows you to float images
\usepackage{latexsym}
%\usepackage{bbding}
%\usepackage {moresize}
\usepackage{listings}
\usepackage{bbding}
\usepackage{blindtext}
\usepackage{hhline}
\usepackage{tikz}
\usetikzlibrary{trees}
%\usetikzlibrary{shapes,backgrounds}
%\usepackage{pgfplots}
%\usetikzlibrary{arrows}
\usepackage{enumitem}
%\doublespacing
%\usepackage{geometry}
\usepackage{amsthm}
\usepackage{color}
%\usepackage{array,multirow}
%\usepackage{subcaption}
%\usepackage{pst-plot}
%	\psset{xunit=15mm}
%\geometry{verbose,tmargin=1in,bmargin=1in,lmargin=.5in,rmargin=.5in}
\setlength{\parskip}{\bigskipamount}
\setlength{\parindent}{0pt}
\usepackage{multicol}

\newenvironment{problem}[3][Problem]{\begin{trivlist}
\item[\hskip \labelsep {\bfseries #1}\hskip \labelsep {\bfseries #2.}]}{\end{trivlist}}

\newcommand{\barr}{\bar{r}}
\newcommand{\ddx}{\frac{d}{dx}}
\newcommand{\infsum}{\sum_{n=1}^{\infty }}

\title{Problem Set 2 \thanks{Problems:3.4,3.7,3.10,3.18,3.20,3.22,3.32}}
\author{Ian McGroarty \\
	Course Number: 555.444}
\date{September 9, 2019}

\begin{document}

\maketitle
\newpage
%%%%%%%%%%%%%%%%%%%%%%%%%%%%%%%%%%%%%%%%%%%%%%%%%%%%%%%%
%%%%%%%%%%%%%%%%%%%%%%%%%%%%%%%%%%%%%%%%%%%%%%%%%%%%%%%%
%%%%%%%%%%%%%%%%%%%%%%%%%%%%%%%%%%%%%%%%%%%%%%%%%%%%%%%%
\begin{problem}{3.4}. Under what circumstances does a minimum-variance hedge portfolio lead to no hedging at all?  \\

It may be the case the minimum variance hedge ratio defined as $h^{*}= \rho \frac{\sigma_s}{\sigma_f} = 0$. This would be the case if the futures prices and the asset prices moved independently of one another. Thus, the coefficient of correlation between the standard deviation of $\Delta S$ and the standard deviation of $\Delta F$, denoted by $\rho = 0. $
  \end{problem}

\begin{problem}{3.7}. A company has a \$20 million portfolio with a beta of 1.2. It would like to use futures contracts
on a stock index to hedge its risk. The index futures is currently standing at 1080, and each
contract is for delivery of \$250 times the index. What is the hedge that minimizes risk? What
should the company do if it wants to reduce the beta of the portfolio to 0.6?  \\

First, we calculate the value of the futures index, $V_F = 250 \cdot 1080 = 270,000$. With $\beta = 1.2$ and the toal value of the portfolio, $V_aA = 20,000,000$, we can find the number of futures that should be shorted using: 
\begin{align*}
 N^{*} & = \beta \frac{V_A}{V_F}  && \text{Eqn. 3.5, pg 64} \\ 
&= 1.2 * \frac{20,000,000}{270,000} = 88.889
\end{align*}
We should round this to 89 shares (since we have to buy a whole number of futures contracts! Should the company want to reduce the beta of the portfolio to 0.6, it should short: 
\begin{align*} 
N &= (\beta - \beta^{*})\frac{V_A}{V_F}  && \text{page 67} \\
&= (1.2-0.6)\cdot \frac{20,000,000}{270,000} = 44.444 
\end{align*}
Meaning, 45 shares should be shorted rather than 89 shares. 
\end{problem} 


\begin{problem}{3.10}. Explain why a short hedger’s position improves when the basis strengthens unexpectedly and
worsens when the basis weakens unexpectedly.  \\

The basis is defined as $b_1 = S_1 - F_1$ and $b_2 = S_2 - F_2$ (page 56). A short position is taken with the profit from the short position = $F_1-F_2$. Adding $F_1$ to both sides of the $b_2$ equation yields: $F_1 + b_2 = S_2 +(F_1-F_2)$. Here we see the hedger's position in period 2, since the asset will sell at $S_2$ and the profit from futures will be $(F_1-F_2)$. A basis is said to strengthen if $b_1 < b_2 \implies S_1 - F_1 < S_2 - F_2$. We can break this down into the two components: If $S_2$ increases relative to $S_1$ the short hedger receives a higher price for the asset. If $F_2$ lative to $_1$ the short hedger will recive a larger profit from the future. Thus, the short hedger is better off if the basis increases unexpectedly. 
\end{problem}

\begin{problem}{3.17}. A corn farmer argues I do not use futures contracts for hedging. My real risk is not the price of
corn. It is that my whole crop gets wiped out by the weather. Discuss this viewpoint. Should the
farmer estimate his or her expected production of corn and hedge to try to lock in a price for
expected production?\\ 

So the key here is to recognize the the price shocks in corn do not happen in a bubble. A weather event likely to affect one farmer is likely to affect all farmers is likely to affect the price of corn. If there is a dust bowl round 2, the price will rise dramatically (\#supplyanddemand) but the farmer might not have any corn to sell anyway. If there is a really great year, the surplus of corn will drive down prices, hurting the farmer, but will also mean a good year in terms of the amount of corn harvested by the farmer. That being said, this is not necessarily the case. Changing technology, regional differences, and market volatility also impact the price of corn independent of the the farmers yield. For example, a new farming technology that allows a foreign country to produce a lot of corn will drive down the global price of corn and hurt the farmer. In this case the framers loss due to the price shock is not offset by the harvest yield, and thus hedging would have been appropriate. 
\end{problem} 
  

\begin{problem}{3.18}. On July 1, an investor holds 50,000 shares of a certain stock. The market price is \$30 per share. The investor is interested in hedging against movements in the market over the next month and decides to use the September Mini S\&P 500 futures contract. The index is currently 1,500 and one contract is for delivery of \$50 times the index. The beta of the stock is 1.3. What strategy should the investor follow? Under what circumstances will it be profitable?   \\

Okay, so the investor will hedge using the Mini S\&P with, $V_F = \$50 \cdot 1500 = 75000$. And the investor hold 50000 shares of a stock valued at \$30/share, $V_A = 30 \cdot 50000 = 1,500,000 $. The number of contracts that should be shorted is: $1.3 \cdot 1500000/75000 = 26$. Suppose the stock falls from \$30 to \$29.50. and the Index falls from 1500 to 1100. The loss on the stock: $0.5 \cdot 50,000 = \$25,000$


\end{problem}

\begin{problem}{3.20}. A futures contract is used for hedging. Explain why the daily settlement of the contract can give rise to cash flow problems. \\

The daily settlement of a contract requires some amount of the change in the underlying asset to be paid out as the value of the contract changes day to day. If there is a large downward shock to the value of the future (e.g. price rise on a short future), the contract holder may run into liquidity trouble needing to settle the lost amount. This is because the margin call requires the investor to pay into the margin account before having received any profit from the future. 
\end{problem}


\begin{problem}{3.22}. Suppose the one-year gold lease rate is 1.5\% and the one-year risk-free rate is 5.0\%. Both rates are compounded annually. Use discussion in Business Snapshot 3.1 to calculate the maximum one-year forward price Goldman Sachs should quote for gold when the spot price is \$1,200.\\

 Hedge by borrowing gold from the central bank, selling it immediately on the spot market for \$1,200, and investing the proceeds at the risk free rate $1200\cdot (1+ 0.05)= 1260$. Pay the lease rate on the gold, $1200 \cdot(0 + .015) = \$18 \ implies (1260-18) = \$1242$.  At the end of the year, you buy the gold from the gold mining company and use it to repay the central bank. If Goldman quotes the one year forward price for gold below \$1242 it will make a profit. 






\end{problem}

\newpage
\begin{problem}{3.32}. A company will buy 1 million pounds of copper. Each contract accounts for 25,000 pounds. The Initial margin = \$2,000/contract. The maintenence margin is \$1,500/contract. They want to hedge 80\% $\implies 1,000,000 \cdot 0.8 = 800,000$ pounds of copper $\implies 800,000/25,000 = 32$ contracts. $\implies 32 \cdot 2,000 = \$ 64,000 $ Initial Margin. 
\begin{table}[h!]
\centering
\begin{tabular}{ l | c | c | c | c | r }	
Date & Oct 2017 & Feb 2018 & Aug 2018 & Feb 2019 & Aug 2019 \\
\hline
Spot Price &  372.00 & 369.00 & 365.00 & 377.00 & 388.00 \\
Mar 2018 Futures Price & 372.30 & 369.10 & & & \\
Sep 2018 Futures Price & 372.80 & 370.20 & 364.80 & & \\
Mar 2019 Futures Price &  & 370.70 & 364.30 & 376.70 & \\ 
Sep 2019 Futures Price & & & 364.20 & 376.50 & 388.20 \\
\end{tabular}
\caption{}
\end{table}

\underline{Planning:} From looking at the table, we can see that the Mar 2018 Future is useless since it looses more money than the Sep 2018 Future going into Feb 2018. Similarly, the Mar 2019 future performs worse than the Sep 2018 future going into Aug 2018. However, the March 2019 future out performs Sep 2019 futures going into Feb 2019.
 So here is the plan. We buy Sep 2018 futures in October 2017, sell them in Aug 2018 and use the funds to purchase March 2019 futures. Sell those in Feb 2019 and use the funds to purchase Sep 2019 funds to be sold in Aug 2019. 

\underline{Answer:} Following the plan above: September 2018 contract is shorted at \$372.80 and closed out in August 2018 for 364.80 for a profit of (372.80-364.80 = \$-8.00). The March 2019 contract is shorted at \$364.30 and closed in February 2019 for \$376.70 for a profit of (376.70 - 364.30= \$12.4). Finally, the September 2019 contract is shorted at \$376.50 and closed at \$388.20 for a profit of (388.20-376.50 = \$11.70). This gives us a profit per pound of, $(-8.00 + 12.40 + 11.70) = 16.1$. The price of copper rose from 372.00 to 388.00 meaning the company would loose \$16 per pound.

See above that the Initial Margin is 64,000 for 800,000 pounds. The loss of \$8/pound translates to a loss of $(8 \cdot 800,000 = \$6,400,000)$ Which would indeed translate to a margin call to ensure liquidity. 	

\end{problem}

\end{document}




% Set the overall layout of the tree




\tikzstyle{level 1}=[level distance=3.5cm, sibling distance=3.5cm]
\tikzstyle{level 2}=[level distance=3.5cm, sibling distance=2cm]

% Define styles for bags and leafs
\tikzstyle{bag} = [text width=4em, text centered]
\tikzstyle{end} = [circle, minimum width=3pt,fill, inner sep=0pt]

\begin{tikzpicture}[grow=right, sloped]
\node[bag] {Bag 1 $4W, 3B$}
    child {
        node[bag] {Bag 2 $4W, 5B$}        
            child {
                node[end, label=right:
                    {$P(W_1\cap W_2)=\frac{4}{7}\cdot\frac{4}{9}$}] {}
                edge from parent
                node[above] {$W$}
                node[below]  {$\frac{4}{9}$}
            }
            child {
                node[end, label=right:
                    {$P(W_1\cap B_2)=\frac{4}{7}\cdot\frac{5}{9}$}] {}
                edge from parent
                node[above] {$B$}
                node[below]  {$\frac{5}{9}$}
            }
            edge from parent 
            node[above] {$W$}
            node[below]  {$\frac{4}{7}$}
    }
    child {
        node[bag] {Bag 2 $3W, 6B$}        
        child {
                node[end, label=right:
                    {$P(B_1\cap W_2)=\frac{3}{7}\cdot\frac{3}{9}$}] {}
                edge from parent
                node[above] {$B$}
                node[below]  {$\frac{3}{9}$}
            }
            child {
                node[end, label=right:
                    {$P(B_1\cap B_2)=\frac{3}{7}\cdot\frac{6}{9}$}] {}
                edge from parent
                node[above] {$W$}
                node[below]  {$\frac{6}{9}$}
            }
        edge from parent         
            node[above] {$B$}
            node[below]  {$\frac{3}{7}$}
    };
\end{tikzpicture}


\section{Definitions}
\underline{Def: Forward Rate Formulas} (pg 79). The implied forward rate between times $t_1$ and $t_2$ is the rate of interset between those times that is consistent with a given spot rate curve. For Yearly compounding, the forward rate is:  
\begin{align*}
f_{i,j} =& [\frac{(1+s_j)^j}{(1+s_i)^i}]^{1/(j-i)}-1 \\
 e^{s(t_2)t_2} =& e^{s(t_1)t_1}e^{f_{t_1,t_2}(t_2-t_1)}
\end{align*}

\underline{Discount Factor Relation} The discount facot between periods i and j is defined as $$ d_{i,j}=[\frac{1}{1+f_{i,j}}]^{j-i}$$ These factors satisfy the compounding rule: $d_{i,k}=d_{i,j}d_{j,k}$\\

\underline{Def. Derivative (Ross pg 223)} Let F be a real valued function defined on an open interval contained a point a. We say f is differentiable at a, or f has derivative at a if the limit $$ f'(a) = \lim_{x \to a} \frac{f(x)-f(a)}{x-a} $$




https://www.investopedia.com/university/advancedbond/bond-pricing.asp
https://quant.stackexchange.com/questions/22288/duration-of-perpetual-bond
http://people.stern.nyu.edu/gyang/foundations/sample-final-solutions.html
http://pages.stern.nyu.edu/~jcarpen0/courses/b403333/07convexh.pdf
https://web.stanford.edu/class/msande247s/2009/summer%2009%20week%205/Bond%20Formula%20Sheet.pdf


\underline{Def: Forward Rate Formulas} (pg 79). The implied forward rate between times $t_1$ and $t_2$ is the rate of interset between those times that is consistent with a given spot rate curve. For Yearly compounding, the forward rate is:  
\begin{align*}
f_{i,j} =& [\frac{(1+s_j)^j}{(1+s_i)^i}]^{1/(j-i)}-1 \\
 e^{s(t_2)t_2} =& e^{s(t_1)t_1}e^{f_{t_1,t_2}(t_2-t_1)}
\end{align*}

\underline{Discount Factor Relation} The discount facot between periods i and j is defined as $$ d_{i,j}=[\frac{1}{1+f_{i,j}}]^{j-i}$$ These factors satisfy the compounding rule: $d_{i,k}=d_{i,j}d_{j,k}$\\

\underline{Def. Derivative (Ross pg 223)} Let F be a real valued function defined on an open interval contained a point a. We say f is differentiable at a, or f has derivative at a if the limit $$ f'(a) = \lim_{x \to a} \frac{f(x)-f(a)}{x-a} $$



\begin{align*}
\text{Maximize  } & 4x_1 +5x_2 +3x_3 +4.3x_4 + x_5 + 1.5x_6 + 2.5x_7 + 0.3x_8 + x_9 + 2x_{10} \\
\text{Subject to } & 2x_1 + 3x_2 + 1.5x_3 + 2.2x_4 +0.5x_5 +15x_6 + 2.5x_7 +0.1x_8 + 0.6x_9 + x_{10} \leq 5 \\ 
& x_1 + x_2 + x_3 + x_4 \leq 1 \\
& x_5 + x_6 + x_7 \leq 1 \\
& x_8 + x_9 + x_{10} \leq 1 \\
\end{align*}