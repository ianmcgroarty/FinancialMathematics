\documentclass[12pt]{article}
\usepackage[T1]{fontenc}
%\usepackage[latin9]{inputenc}
\usepackage[utf8]{inputenc}
\usepackage[english]{babel}
\usepackage{amsmath}
\usepackage{amsfonts}
\usepackage{amssymb}
%\usepackage{setspace}
\usepackage{rotating}
\usepackage{graphics}
\usepackage{eurosym}
\usepackage[round]{natbib}
%\usepackage{graphicx}
%\usepackage{float} 				%allows you to float images
\usepackage{latexsym}
%\usepackage{bbding}
%\usepackage {moresize}
\usepackage{listings}
\usepackage{bbding}
\usepackage{blindtext}
\usepackage{hhline}
\usepackage{tikz}
\usetikzlibrary{trees}
%\usetikzlibrary{shapes,backgrounds}
%\usepackage{pgfplots}
%\usetikzlibrary{arrows}
\usepackage{enumitem}
\doublespacing
%\usepackage{geometry}
\usepackage{amsthm}
\usepackage{color}
%\usepackage{array,multirow}
%\usepackage{subcaption}
%\usepackage{pst-plot}
%	\psset{xunit=15mm}
%\geometry{verbose,tmargin=1in,bmargin=1in,lmargin=.5in,rmargin=.5in}
\setlength{\parskip}{\bigskipamount}
\setlength{\parindent}{0pt}
\usepackage{multicol}

\newenvironment{problem}[3][Problem]{\begin{trivlist}
\item[\hskip \labelsep {\bfseries #1}\hskip \labelsep {\bfseries #2.}]}{\end{trivlist}}

\newcommand{\barr}{\bar{r}}
\newcommand{\ddx}{\frac{d}{dx}}
\newcommand{\infsum}{\sum_{n=1}^{\infty }}

\title{Problem Set 10 \thanks{Problems:4,5,6,7,10}}
\author{Ian McGroarty \\
	Course Number: 625.641}
\date{August 7, 2019}

\begin{document}

\maketitle

Sorry in advance. Its been a week, and I didn't really have time to do this. I've included photos of the white board on which I work these out. Let me know if you have any problems. Again sorry. 
\newpage
%%%%%%%%%%%%%%%%%%%%%%%%%%%%%%%%%%%%%%%%%%%%%%%%%%%%%%%%
%%%%%%%%%%%%%%%%%%%%%%%%%%%%%%%%%%%%%%%%%%%%%%%%%%%%%%%%
%%%%%%%%%%%%%%%%%%%%%%%%%%%%%%%%%%%%%%%%%%%%%%%%%%%%%%%%
\begin{problem}{4}.ss \\
hics[width=\textwidth ]{mod10p10c.png}

  \end{problem}
  
\end{document}




% Set the overall layout of the tree




\tikzstyle{level 1}=[level distance=3.5cm, sibling distance=3.5cm]
\tikzstyle{level 2}=[level distance=3.5cm, sibling distance=2cm]

% Define styles for bags and leafs
\tikzstyle{bag} = [text width=4em, text centered]
\tikzstyle{end} = [circle, minimum width=3pt,fill, inner sep=0pt]

\begin{tikzpicture}[grow=right, sloped]
\node[bag] {Bag 1 $4W, 3B$}
    child {
        node[bag] {Bag 2 $4W, 5B$}        
            child {
                node[end, label=right:
                    {$P(W_1\cap W_2)=\frac{4}{7}\cdot\frac{4}{9}$}] {}
                edge from parent
                node[above] {$W$}
                node[below]  {$\frac{4}{9}$}
            }
            child {
                node[end, label=right:
                    {$P(W_1\cap B_2)=\frac{4}{7}\cdot\frac{5}{9}$}] {}
                edge from parent
                node[above] {$B$}
                node[below]  {$\frac{5}{9}$}
            }
            edge from parent 
            node[above] {$W$}
            node[below]  {$\frac{4}{7}$}
    }
    child {
        node[bag] {Bag 2 $3W, 6B$}        
        child {
                node[end, label=right:
                    {$P(B_1\cap W_2)=\frac{3}{7}\cdot\frac{3}{9}$}] {}
                edge from parent
                node[above] {$B$}
                node[below]  {$\frac{3}{9}$}
            }
            child {
                node[end, label=right:
                    {$P(B_1\cap B_2)=\frac{3}{7}\cdot\frac{6}{9}$}] {}
                edge from parent
                node[above] {$W$}
                node[below]  {$\frac{6}{9}$}
            }
        edge from parent         
            node[above] {$B$}
            node[below]  {$\frac{3}{7}$}
    };
\end{tikzpicture}


\section{Definitions}
\underline{Def: Forward Rate Formulas} (pg 79). The implied forward rate between times $t_1$ and $t_2$ is the rate of interset between those times that is consistent with a given spot rate curve. For Yearly compounding, the forward rate is:  
\begin{align*}
f_{i,j} =& [\frac{(1+s_j)^j}{(1+s_i)^i}]^{1/(j-i)}-1 \\
 e^{s(t_2)t_2} =& e^{s(t_1)t_1}e^{f_{t_1,t_2}(t_2-t_1)}
\end{align*}

\underline{Discount Factor Relation} The discount facot between periods i and j is defined as $$ d_{i,j}=[\frac{1}{1+f_{i,j}}]^{j-i}$$ These factors satisfy the compounding rule: $d_{i,k}=d_{i,j}d_{j,k}$\\

\underline{Def. Derivative (Ross pg 223)} Let F be a real valued function defined on an open interval contained a point a. We say f is differentiable at a, or f has derivative at a if the limit $$ f'(a) = \lim_{x \to a} \frac{f(x)-f(a)}{x-a} $$




https://www.investopedia.com/university/advancedbond/bond-pricing.asp
https://quant.stackexchange.com/questions/22288/duration-of-perpetual-bond
http://people.stern.nyu.edu/gyang/foundations/sample-final-solutions.html
http://pages.stern.nyu.edu/~jcarpen0/courses/b403333/07convexh.pdf
https://web.stanford.edu/class/msande247s/2009/summer%2009%20week%205/Bond%20Formula%20Sheet.pdf


\underline{Def: Forward Rate Formulas} (pg 79). The implied forward rate between times $t_1$ and $t_2$ is the rate of interset between those times that is consistent with a given spot rate curve. For Yearly compounding, the forward rate is:  
\begin{align*}
f_{i,j} =& [\frac{(1+s_j)^j}{(1+s_i)^i}]^{1/(j-i)}-1 \\
 e^{s(t_2)t_2} =& e^{s(t_1)t_1}e^{f_{t_1,t_2}(t_2-t_1)}
\end{align*}

\underline{Discount Factor Relation} The discount facot between periods i and j is defined as $$ d_{i,j}=[\frac{1}{1+f_{i,j}}]^{j-i}$$ These factors satisfy the compounding rule: $d_{i,k}=d_{i,j}d_{j,k}$\\

\underline{Def. Derivative (Ross pg 223)} Let F be a real valued function defined on an open interval contained a point a. We say f is differentiable at a, or f has derivative at a if the limit $$ f'(a) = \lim_{x \to a} \frac{f(x)-f(a)}{x-a} $$



\begin{align*}
\text{Maximize  } & 4x_1 +5x_2 +3x_3 +4.3x_4 + x_5 + 1.5x_6 + 2.5x_7 + 0.3x_8 + x_9 + 2x_{10} \\
\text{Subject to } & 2x_1 + 3x_2 + 1.5x_3 + 2.2x_4 +0.5x_5 +15x_6 + 2.5x_7 +0.1x_8 + 0.6x_9 + x_{10} \leq 5 \\ 
& x_1 + x_2 + x_3 + x_4 \leq 1 \\
& x_5 + x_6 + x_7 \leq 1 \\
& x_8 + x_9 + x_{10} \leq 1 \\
\end{align*}