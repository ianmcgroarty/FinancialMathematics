\documentclass[12pt]{article}
\usepackage[T1]{fontenc}
%\usepackage[latin9]{inputenc}
\usepackage[utf8]{inputenc}
\usepackage[english]{babel}
\usepackage{amsmath}
\usepackage{amsfonts}
\usepackage{amssymb}
\usepackage{setspace}
\usepackage{rotating}
\usepackage{graphics}
\usepackage[round]{natbib}
%\usepackage{graphicx}
%\usepackage{float} 				%allows you to float images
\usepackage{latexsym}
\usepackage{bbding}
%\usepackage {moresize}
\usepackage{listings}
\usepackage{bbding}
\usepackage{blindtext}
\usepackage{hhline}
%\usepackage{tikz}
%\usetikzlibrary{shapes,backgrounds}
%\usepackage{pgfplots}
%\usetikzlibrary{arrows}
\usepackage{enumitem}
\doublespacing
%\usepackage{geometry}
\usepackage{amsthm}
\usepackage{color}
%\usepackage{array,multirow}
%\usepackage{subcaption}
%\usepackage{pst-plot}
%	\psset{xunit=15mm}
%\geometry{verbose,tmargin=1in,bmargin=1in,lmargin=.5in,rmargin=.5in}
\setlength{\parskip}{\bigskipamount}
\setlength{\parindent}{0pt}
\usepackage{multicol}

\newenvironment{problem}[2][Problem]{\begin{trivlist}
\item[\hskip \labelsep {\bfseries #1}\hskip \labelsep {\bfseries #2.}]}{\end{trivlist}}

\title{Problem Set 1 \thanks{Problems 4,5,8,13,14 (pg 38-40)}}
\author{Ian McGroarty \\
	Course Number: 625.641}
\date{June 4, 2019}

\begin{document}

\maketitle
\newpage
%%%%%%%%%%%%%%%%%%%%%%%%%%%%%%%%%%%%%%%%%%%%%%%%%%%%%%%%
%%%%%%%%%%%%%%%%%%%%%%%%%%%%%%%%%%%%%%%%%%%%%%%%%%%%%%%%%%%%%%%%%%%%%%%%%%%%%%%%%%%%%%%%%%%%%%%%%%%%%%%%%%%%%%%%
%%%%%%%%%%%%%%%%%%%%%%%%%%%%%%%%%%%%%%%%%%%%%%%%%%%%%%%%%%%%%%%%%%%%%%%%%%%%%%%%%%%%%%%%%%%%%%%%%%%%%%%%%%%%%%%%

\begin{problem}{2.4} Let $ f(\lambda ) = -1 + \lambda + \lambda^2 $ And define: $ \lambda_{k+1} = \lambda_k - \frac{f(\lambda_k)}{f'(\lambda_k)}$. \\
\textbf{Solution}: First we find that $f'(\lambda) = 1+2\lambda $. Beginning with $\lambda_0 = 1$ we have: 
\begin{align*} 
\lambda_0 = 1 \rightarrow \lambda_{1} &= 1- \frac{-1+1+1^2}{1+2(1)} \\
						\lambda_{1}	&= 1/3 \\ 
\lambda_1 = 1/3 \rightarrow  \lambda_{2} &= 1- \frac{-1+(1/3)+(1/3)^2}{1+2(1/3)} \\
						\lambda_{2}	&= -1/3 \\
\lambda_2 = -1/3 \rightarrow  \lambda_{3} &= 1- \frac{-1+(-1/3)+(-1/3)^2}{1+2(-1/3)} \\
					\lambda_{3} &= -11/3 \\
\lambda_3 = -11/3 \rightarrow  \lambda_{4} &= 1- \frac{-1+(-11/3)+(-11/3)^2}{1+2(-11/3)} \\
					\lambda_{4}&= -1.385 
\end{align*}
\end{problem}
%%%%%%%%%%%%%%%%%%%%%%%%%%%%%%%%%%%%%%%%%%%%%%%%%%%%%
%%%%%%%%%%%%%%%%%%%%%%%%%%%%%%%%%%%%%%%%%%%%%%%%%%%%%
%%%%%%%%%%%%%%%%%%%%%%%%%%%%%%%%%%%%%%%%%%%%%%%%%%%%%

\begin{problem}{2.5} To simplify the cash flow is (-1,0...0,n) where n is the number of years we wait. We can use the net present value formula: 
$$\frac{-1}{1.1^0} + 0 + \cdots + \frac{n}{1.1^{n-1}}$$
We can see that the term the will determine how many years is optimal is the last term, since all else is equal. So basically we want to find the maximum of $ \frac{n}{1.1^{n-1}}$. Well since I am lazy I will use wolfram alpha to compute the critical point and we get 10.46 years. Waiting 10.46 years (not sure if I can sell halfway through the year in this example otherwise 10 years will give us the maximum) will optimize the growth of the tree and the growth of our dollar!
\end{problem} 
%%%%%%%%%%%%%%%%%%%%%%%%%%%%%%%%%%%%%%%%%%%%%%%%%%%%%
%%%%%%%%%%%%%%%%%%%%%%%%%%%%%%%%%%%%%%%%%%%%%%%%%%%%%
%%%%%%%%%%%%%%%%%%%%%%%%%%%%%%%%%%%%%%%%%%%%%%%%%%%%%

\begin{problem}{2.8} AHHH tricky tricky... Well since the \$1,000 deposit is nonrefundable this will not fact0r into the calculations. Assuming that they are indifferent between the quality of the two apartments, they should go with the cheaper option regardless of interest rate or time span. However, if they plan to stay 1 year they should make sure to get a 12 month lease, because terms are subject to change haha! 
\end{problem} 
%%%%%%%%%%%%%%%%%%%%%%%%%%%%%%%%%%%%%%%%%%%%%%%%%%%%%
%%%%%%%%%%%%%%%%%%%%%%%%%%%%%%%%%%%%%%%%%%%%%%%%%%%%%
%%%%%%%%%%%%%%%%%%%%%%%%%%%%%%%%%%%%%%%%%%%%%%%%%%%%%

\begin{problem}{2.13} Compare the payment streams: A = (-100,30,30,30,30,30) and B = (-150,42,42,42,42,42): 
\begin{align*}
NPV_{A} &= \sum^n_{k=0} \frac{x_k}{(1+r)^k} = \frac{-100}{1.05^0} +\frac{30}{1.05^1} \cdots \frac{30}{1.05^4} =29.88  \\
NPV_{B} &= \sum^n_{k=0} \frac{x_k}{(1+r)^k} = \frac{-150}{1.05^0} +\frac{42}{1.05^1} \cdots \frac{42}{1.05^4} = 31.83 \\
IRR_A &\rightarrow 0 =-100 +30c +30c^2 +30c^3 +30c^4 +30c^5 \\
&\rightarrow c = 0.867 \text{ and } r=\frac{1}{c}-1=0.152 \\
IRR_B &\rightarrow 0 =-150 +42c +42c^2 +42c^3 +42c^4 +42c^5 \\
&\rightarrow c = 0.8899 \text{ and }  r=\frac{1}{c}-1=0.124 
\end{align*}
Project B has a higher NPV, but project A has a higher R. The net present value formula recommends B because you get more money more quicky, $42 > 30$. This means that you can earn higher interest faster. But B also has a higher upfront cost than A. So A would needs to make up less money in the same time frame. This happens due to an ``inherent reinvestment assumption'' assuming that cash flows will be reinvested at the corresponding interest rate. For the NPV this is .05 but IRR it is the corresponding R.\footnote{https://financetrain.com/conflict-between-npv-and-irr/} 
\end{problem}

%%%%%%%%%%%%%%%%%%%%%%%%%%%%%%%%%%%%%%%%%%%%%%%%%%%%%
%%%%%%%%%%%%%%%%%%%%%%%%%%%%%%%%%%%%%%%%%%%%%%%%%%%%%
%%%%%%%%%%%%%%%%%%%%%%%%%%%%%%%%%%%%%%%%%%%%%%%%%%%%%
\newpage


\begin{problem}{2.14} Suppose two competing projects have cash flows of the form $(-A_1,B_1,...,B_1)$ and $(A_2, B_2,...,B_2)$. Suppose $B_1/A_1 > B_2/A_2 $. Show that project 1 will have a higher IRR than project 2.
\begin{proof}\footnote{Ross, K. Elementary Analysis, the Theory of Calculus. Spinger, New York. 2013.} For short hand let $\kappa_1 = (c_1 +c_1^2 \cdots c_1^n)$ and $\kappa_2 = (c_2 +c_2^2 \cdots c_2^n)$
\begin{align*}
0&= -A_1 + B_1c_1 +B_1c_1^2 + \cdots + B_1c_1^n \\
&= -A_1 + B_1(c_1 +c_1^2 \cdots c_1^n) \\
&= \frac{-A_1}{B_1} + \kappa_1 && B_1 \neq 0  \\ 
0&= -A_2 + B_2c_2 +B_2c_2^2 + \cdots + B_2c_2^n \\
&= -A_2 + B_2(c_2 +c_2^2 \cdots c_2^n) \\
&= \frac{-A_2}{B_2} + \kappa_2 && B_2 \neq 0  \\ 
\frac{-A_1}{B_1} + \kappa_1 &= \frac{-A_2}{B_2} + \kappa_2 \\ 
\frac{-A_1}{B_1} &= \frac{-A_2}{B_2} + \kappa_2 -\kappa_1 &&\text{Equation 1}\\ 
\text{Since  } \frac{A_1}{B_1} > \frac{A_2}{B_2} & \implies   \frac{-A_1}{B_1} > \frac{-A_2}{B_2} &&\text{Theorem 3.2 (pg 16)}^2 \\ 
 \frac{-A_2}{B_2} + \kappa_2 -\kappa_1 &>  \frac{-A_2}{B_2} && \text{Substitute equation 1 }\\
\kappa_1 < \kappa_2 &\implies c_1 < c_2
\end{align*} 
Since: $r=(1/c)-1$ is decreasing in c,  $c_1<c_2 \implies r_1 >r_2$. 
\end{proof}

\end{problem}


\end{document}


