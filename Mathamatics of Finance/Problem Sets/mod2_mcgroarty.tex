\documentclass[12pt]{article}
\usepackage[T1]{fontenc}
%\usepackage[latin9]{inputenc}
\usepackage[utf8]{inputenc}
\usepackage[english]{babel}
\usepackage{amsmath}
\usepackage{amsfonts}
\usepackage{amssymb}
\usepackage{setspace}
\usepackage{rotating}
\usepackage{graphics}
\usepackage[round]{natbib}
%\usepackage{graphicx}
%\usepackage{float} 				%allows you to float images
\usepackage{latexsym}
\usepackage{bbding}
%\usepackage {moresize}
\usepackage{listings}
\usepackage{bbding}
\usepackage{blindtext}
\usepackage{hhline}
%\usepackage{tikz}
%\usetikzlibrary{shapes,backgrounds}
%\usepackage{pgfplots}
%\usetikzlibrary{arrows}
\usepackage{enumitem}
\doublespacing
%\usepackage{geometry}
\usepackage{amsthm}
\usepackage{color}
%\usepackage{array,multirow}
%\usepackage{subcaption}
%\usepackage{pst-plot}
%	\psset{xunit=15mm}
%\geometry{verbose,tmargin=1in,bmargin=1in,lmargin=.5in,rmargin=.5in}
\setlength{\parskip}{\bigskipamount}
\setlength{\parindent}{0pt}
\usepackage{multicol}

\newenvironment{problem}[2][Problem]{\begin{trivlist}
\item[\hskip \labelsep {\bfseries #1}\hskip \labelsep {\bfseries #2.}]}{\end{trivlist}}

\title{Problem Set 2 \thanks{Problems 1,2,3,4,5}}
\author{Ian McGroarty \\
	Course Number: 625.641}
\date{June 11, 2019}

\begin{document}

\maketitle
\newpage
%%%%%%%%%%%%%%%%%%%%%%%%%%%%%%%%%%%%%%%%%%%%%%%%%%%%%%%%
%%%%%%%%%%%%%%%%%%%%%%%%%%%%%%%%%%%%%%%%%%%%%%%%%%%%%%%%%%%%%%%%%%%%%%%%%%%%%%%%%%%%%%%%%%%%%%%%%%%%%%%%%%%%%%%%
%%%%%%%%%%%%%%%%%%%%%%%%%%%%%%%%%%%%%%%%%%%%%%%%%%%%%%%%%%%%%%%%%%%%%%%%%%%%%%%%%%%%%%%%%%%%%%%%%%%%%%%%%%%%%%%%

\begin{problem}{1}  Find the price of a 10\% bond with 30 years maturity and 5\% yield. \\
\textbf{Solution} For this question we will use the \underline{bond price formula} (3.2 pg 53). We will have to make some assumptions about the bond though: I assume semi-annual payments, and a par value of 100, this will express the price as a percentage of par. With payments made twice a year ($m=2$) for 30 years, we can say that $n=60$. By assuming a par value of 100 $F=100$, the coupon payment is $10\%$ so $(C=100*.1=10)$.  And a 5\% yield means $\lambda = 0.05$
\begin{align*}
P&= \frac{F}{[1+(\lambda / m)]^n} + \frac{C}{\lambda }\{ 1- \frac{1}{[1+(\lambda / m)]^n}\}\\
&= \frac{100}{[1+(0.05/2)]^60} + \frac{10}{0.05}\{1- \frac{1}{[1+(0.05/2)]^60} \} \\ 
Price &= 20. 
\end{align*}
A price of 20 reflects that the price is 20\% of the par value of the bond. 
\end{problem}
%%%%%%%%%%%%%%%%%%%%%%%%%%%%%%%%%%%%%%%%%%%%%%%%%%%%%%%%
%%%%%%%%%%%%%%%%%%%%%%%%%%%%%%%%%%%%%%%%%%%%%%%%%%%%%%%%%%%%%%%%%%%%%%%%%%%%%%%%%%%%%%%%%%%%%%%%%%%%%%%%%%%%%%%%
%%%%%%%%%%%%%%%%%%%%%%%%%%%%%%%%%%%%%%%%%%%%%%%%%%%%%%%%%%%%%%%%%%%%%%%%%%%%%%%%%%%%%%%%%%%%%%%%%%%%%%%%%%%%%%%%

\begin{problem}{2}  Find the present value of the perpetual cash flow stream $(C_0,C_1,...,C_n,...)$ where the n-th year payment $C_n = n-1$ and the prevailing rate is 2\%  \\
 \underline{Def: Perpetual Annuity Formula (pg. 45)} The present value P of a perpetual annuity that pays an amount A every period, beginning one period from the present is $$ P = \frac{A}{r}$$

\textbf{Solution} I really don't understand what you mean by $C_n = n-1$. I figure it has something to do with time indexing but I honestly don't see how they relate since it is a payment. On the other hand, if n is literally just the position of the payment, so the cash flow stream is $(-1,0,1,2,3,4...)$ So it is a growing perpetual annuity. I couldn't find a formula to calculate this but maybe you could just use 1 (since it is infinite it'll just be 1+1, 1+1+1 idk). This would be $\frac{1}{0.02} = 50. $
\end{problem}

%%%%%%%%%%%%%%%%%%%%%%%%%%%%%%%%%%%%%%%%%%%%%%%%%%%%%%%%
%%%%%%%%%%%%%%%%%%%%%%%%%%%%%%%%%%%%%%%%%%%%%%%%%%%%%%%%%%%%%%%%%%%%%%%%%%%%%%%%%%%%%%%%%%%%%%%%%%%%%%%%%%%%%%%%
%%%%%%%%%%%%%%%%%%%%%%%%%%%%%%%%%%%%%%%%%%%%%%%%%%%%%%%%%%%%%%%%%%%%%%%%%%%%%%%%%%%%%%%%%%%%%%%%%%%%%%%%%%%%%%%%
\begin{problem}{3} Compute the duration and the modified duration of a perpetual annuity that pays \$10,000 at the beginning of each year, with the first such payment being 1 year from now. Assume the prevailing rate  \%3 is compounded yearly. \\
 \underline{Def: Perpetual Annuity Formula (pg. 45)} The present value P of a perpetual annuity that pays an amount A every period, beginning one period from the present is $$ P = \frac{A}{r}$$

\underline{Def: Modified Duration (pg. 60)} The derivative of price P with respect to yeild $\lambda $ of a fixed income security is: $ \frac{dP}{d\lambda } = - D_MP $ where $D_M = D/[1+(\lambda / m )$ is the modified duration. \\

\textbf{Solution} First we find $\frac{dP}{d\lambda} = d(c\lambda^{-1}) = -\frac{c}{\lambda^2}$. By definition we know $P=c/\lambda $. Substituting these into the definition of modified duration: 
$$D_M = -\frac{dP}{d\lambda }\cdot \frac{1}{P} =-(-\frac{c}{\lambda^2} \cdot \frac{1}{c/\lambda })=\frac{1}{\lambda} $$
We can then calculate the Macaulay Duration following the note in the definition for modified duration: 
$$D_M = \frac{D}{[1+(\lambda / m )]} \implies  \frac{1}{\lambda } = \frac{D}{1+(\lambda /m)} \implies D = \frac{1+(\lambda /m) } {\lambda }$$ 
The problem states that the coupon payments are annual ($m=1$), the duration calculations do not depend on the coupon payment amount, and the prevailing interest rate $(\lambda = 0.03)$. So $D_M = 1/0.03 = 33.33$ and $D = 34.33$. I might have done something wrong because, I would've expected them to be basically equal. I mean I guess they are fairly close, but the perpetual annuity is forever, so shouldn't the modified duration be extremely close to the Macaulay duration? 
\end{problem}

%%%%%%%%%%%%%%%%%%%%%%%%%%%%%%%%%%%%%%%%%%%%%%%%%%%%%%%%
%%%%%%%%%%%%%%%%%%%%%%%%%%%%%%%%%%%%%%%%%%%%%%%%%%%%%%%%%%%%%%%%%%%%%%%%%%%%%%%%%%%%%%%%%%%%%%%%%%%%%%%%%%%%%%%%
%%%%%%%%%%%%%%%%%%%%%%%%%%%%%%%%%%%%%%%%%%%%%%%%%%%%%%%%%%%%%%%%%%%%%%%%%%%%%%%%%%%%%%%%%%%%%%%%%%%%%%%%%%%%%%%%

\begin{problem}{4} Fine the convexity of a zero coupon bond in terms of the bond price P, maturity T, yield $\lambda $ and number of periods per year $m$. Find the limit as $m\rightarrow \infty $. \\
\underline{Bond Price Formula (pg 53)} The price of a bond, having exactly n coupon periods remaining to maturity and a yield to maturity of $\lambda $ satisfies: 
$$P= \frac{F}{[1+(\lambda / m)]^t} + \frac{C}{\lambda }\{ 1- \frac{1}{[1+(\lambda / m)]^t}\}$$

\underline{Convexity formula (pg 66)} Convexity is the value of C defined as: $$ C = \frac{1}{P}\frac{d^2P}{d\lambda^2}$$
\textbf{Solution} Well since the coupon payment is 0 the price is just $P= \frac{F}{[1+(\lambda / m)]^n}$ I will use wolfram alpha to calculate: 
\begin{align*}
\frac{d^2P}{d\lambda^2} &= \frac{mt(mt+1)}{(1+\lambda /m)^{mt}(m+\lambda )^2} \\ 
C &= (1+\lambda /m)^{mt} \cdot \frac{mt(mt+1)}{(1+\lambda /m)^{mt}(m+\lambda )^2} \\
&=  \frac{mt(mt+1)}{(m+\lambda )^2} \\
lim_{m\rightarrow \infty }& \frac{m^2t^2 + mt}{m^2 + 2m\lambda + \lambda^2} && \text{take the limit}\\
& \frac{t^2 + t/m}{1 + 2\lambda /m + \frac{\lambda^2}{m^2}} && \text{divide by $m^2$} \\
lim_{m\rightarrow \infty }& (2\lambda /m), \ (\frac{\lambda^2}{m^2}), \ (t/m) \rightarrow 0 \\
\implies & lim_{m\rightarrow \infty } C = t^2
\end{align*}
\end{problem}

%%%%%%%%%%%%%%%%%%%%%%%%%%%%%%%%%%%%%%%%%%%%%%%%%%%%%%%%
%%%%%%%%%%%%%%%%%%%%%%%%%%%%%%%%%%%%%%%%%%%%%%%%%%%%%%%%%%%%%%%%%%%%%%%%%%%%%%%%%%%%%%%%%%%%%%%%%%%%%%%%%%%%%%%%
%%%%%%%%%%%%%%%%%%%%%%%%%%%%%%%%%%%%%%%%%%%%%%%%%%%%%%%%%%%%%%%%%%%%%%%%%%%%%%%%%%%%%%%%%%%%%%%%%%%%%%%%%%%%%%%%


\begin{problem}{5} Consider a 5\% callable bond with 20 years maturity and 8\% yield which pays the face value plus 10\% if it is redeemed before maturity If after 10 years the bond is redeemed, find an upper bound for the yield at that time. Assume that coupon payments are made once per years. 

\underline{Def: Callable Bonds} A bond is callable if the issuer has the right to repurchase the bond at a specified price. Usually this call price falls with time, and often there is an initial call protection period where the bond can not be called.

\textbf{Solution} I'm not going to keep writing out these formulas since I am running short on time! But I use the bond price formula (pg 53, exampled above) using a face value of \$1,000 a coupon payment of \$50 the $\lambda = 0.08$, annula payments m=1, for n=20 years. I wrote a function in R to do this (though one probably already existed) and the Price = \$ 705.45. Since we are redeeming before maturity we receive the face value plus 10\% is \$1,100.  And then since theres only 10 years left we use n=10. Then we need to calculate the interest rate that could be used to make it so that the $$ P = 705.45 \approx \frac{1100}{[1+\lambda]^{10}} + \frac{50}{\lambda }\{ 1- \frac{1}{[1+\lambda ]^{10}}\}$$
I'm not sure how I was supposed to do it, but basically I just tried a bunch of different values using my R formula and it has to be less that \%10.05. 

\end{problem}
\end{document}

https://www.investopedia.com/university/advancedbond/bond-pricing.asp
https://quant.stackexchange.com/questions/22288/duration-of-perpetual-bond
http://people.stern.nyu.edu/gyang/foundations/sample-final-solutions.html
http://pages.stern.nyu.edu/~jcarpen0/courses/b403333/07convexh.pdf
https://web.stanford.edu/class/msande247s/2009/summer%2009%20week%205/Bond%20Formula%20Sheet.pdf