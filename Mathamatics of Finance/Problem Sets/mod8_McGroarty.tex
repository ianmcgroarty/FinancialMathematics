\documentclass[12pt]{article}
\usepackage[T1]{fontenc}
%\usepackage[latin9]{inputenc}
\usepackage[utf8]{inputenc}
\usepackage[english]{babel}
\usepackage{amsmath}
\usepackage{amsfonts}
\usepackage{amssymb}
\usepackage{setspace}
\usepackage{rotating}
\usepackage{graphics}
\usepackage[round]{natbib}
%\usepackage{graphicx}
%\usepackage{float} 				%allows you to float images
\usepackage{latexsym}
\usepackage{bbding}
%\usepackage {moresize}
\usepackage{listings}
\usepackage{bbding}
\usepackage{blindtext}
\usepackage{hhline}
\usepackage{tikz}
\usetikzlibrary{trees}
%\usetikzlibrary{shapes,backgrounds}
%\usepackage{pgfplots}
%\usetikzlibrary{arrows}
\usepackage{enumitem}
\doublespacing
%\usepackage{geometry}
\usepackage{amsthm}
\usepackage{color}
%\usepackage{array,multirow}
%\usepackage{subcaption}
%\usepackage{pst-plot}
%	\psset{xunit=15mm}
%\geometry{verbose,tmargin=1in,bmargin=1in,lmargin=.5in,rmargin=.5in}
\setlength{\parskip}{\bigskipamount}
\setlength{\parindent}{0pt}
\usepackage{multicol}

\newenvironment{problem}[3][Problem]{\begin{trivlist}
\item[\hskip \labelsep {\bfseries #1}\hskip \labelsep {\bfseries #2.}]}{\end{trivlist}}

\newcommand{\barr}{\bar{r}}
\newcommand{\ddx}{\frac{d}{dx}}
\newcommand{\infsum}{\sum_{n=1}^{\infty }}

\title{Problem Set 8 \thanks{Problems:3,4,5,6,9}}
\author{Ian McGroarty \\
	Course Number: 625.641}
\date{July 22, 2019}

\begin{document}

\maketitle
\newpage
%%%%%%%%%%%%%%%%%%%%%%%%%%%%%%%%%%%%%%%%%%%%%%%%%%%%%%%%
%%%%%%%%%%%%%%%%%%%%%%%%%%%%%%%%%%%%%%%%%%%%%%%%%%%%%%%%
%%%%%%%%%%%%%%%%%%%%%%%%%%%%%%%%%%%%%%%%%%%%%%%%%%%%%%%%
\begin{problem}{3 (risk aversion)}.  Suppose $U(x)$ is a utility function with arrow-pratt risk aversion coefficient $a(x)$. Let $V(x) = a + bU(x)$ . Find the risk aversion coefficient for $V(x)$\\
\textbf{Solution} $$ a(x) = \frac{U''(x)}{U(x)} \; \text{Definition of Risk Aversion Coefficient (pg 233)}$$
So $V'(x) = \ddx a + b(U(x)) = b\cdot U'(x)$ and $V''(x) = \ddx  b\cdot U'(x) =  b\cdot U''(x)$ Thus $a_V(x) = \frac{V''(x)}{V(x)} = \frac{bU''(x)}{bU(x)} = a(x)$
\end{problem}

\begin{problem}{4 Relative Risk Aversion}. The Arrow-Pratt relative risk aversion coefficient is $$ \mu (x) = \frac{xU''(x)}{U'(x)}$$ Show that the utility function $U(x) = ln x$ and $U(x) = \gamma x^\gamma $ have constant relative risk aversion coefficients. \\
\textbf{Solution} 
\begin{align*}
 \ddx ln(x) = 1/x \; \& \; \ddx 1/x = -1/x^2 &\implies \mu (x) = \frac{-1/x^2 \cdot x}{1/x} = \frac{-1/x}{1/x} = -1  \\
 \ddx yx^y = y^2x^{y-1} \; \& \; \ddx  y^2x^{y-1} =  y^2(y-1)x^{y-2} &\implies \mu (x) = \frac{ y^2(y-1)x^{y-2} \cdot x}{y^2x^{y-1} } = \\
  &= \frac{ y^2(y-1)x^{y-1}}{y^2x^{y-1} } = (y-1) 
\end{align*}
\end{problem}

\begin{problem}{(5) Equivalence}: Utility function $U(x)$ over $A\leq x \leq B $. $U(A)=A$ and $U(B) = B$. 
			Equivalent utility function $V(x)$ over $A'\leq x \leq B' $. $V(A')=A'$ and $V(B') = B'$.
$V(x) = aU(x) +b $ Find $a,b$. \\
\textbf{Solution} \\
\begin{align*}
V(x) &= aU(x) + b \implies V(A') = aU(A') + b \\
&\implies V(A')-aU(A') = A' - aU(A')= b =  B'-aU(B') \\
&\implies a = \frac{A' - B'}{U(A') - U(B')} \\
&\implies V(x) =  \frac{A' - B'}{U(A') - U(B')}U(x) + b \\
&\implies V(A')- \frac{A' - B'}{U(A') - U(B')}\cdot U(A') = b \\
&\implies b = A'- \frac{A' - B'}{U(A') - U(B')}\cdot U(A')
\end{align*}

			
\end{problem}

%%%%%%%%%%%%%%%%%%%%%%%%%%%%%%%%%%%%%%%%%%%%%%%%%%%%%%%%
%%%%%%%%%%%%%%%%%%%%%%%%%%%%%%%%%%%%%%%%%%%%%%%%%%%%%%%%
%%%%%%%%%%%%%%%%%%%%%%%%%%%%%%%%%%%%%%%%%%%%%%%%%%%%%%%%
\begin{problem}{6 (HARA)}. The HARA class of utility functions is defined by: 
$$ U(x) = \frac{1-\gamma }{\gamma } (\frac{ax}{1-\gamma } + b)^{\gamma } $$ 
Show how the parameters $\gamma , a, b $ can be chosen to represent: 

\textbf{(a) Linear}: Let $b\rightarrow 0, \gamma \rightarrow 1, a = 1$:
\begin{align*}
U(x) &= \frac{1-\gamma }{\gamma } (\frac{ax}{1-\gamma } + b)^{\gamma } = \frac{1-\gamma }{\gamma } (\frac{x}{1-\gamma } )^{\gamma } \\
&lim_{\gamma \rightarrow 1}x^{\gamma} = x \\
\implies&  lim_{\gamma \rightarrow 1} U(x) = \frac{1-\gamma }{\gamma } (\frac{x}{1-\gamma } ) = \frac{x}{\gamma } = x \\ 
\end{align*}
\textbf{(b) Quadratic}: Let $\gamma = 2, a=2, b=1$: 
\begin{align*}
U(x) &= \frac{1-\gamma }{\gamma } (\frac{ax}{1-\gamma } + b)^{\gamma } = \frac{1-2 }{2 } (\frac{2x}{1-2 } )^{2} \\
&= -1/2[-x+1]^2 = -1/2[x^2 - 2x + 1] = -1/2x^2 + x - 1/2 
\end{align*}

\textbf{(c) Exponential} $Let \gamma \rightarrow -\infty , b=0$:
\begin{align*}
U(x) &= \frac{1-\gamma }{\gamma } (\frac{ax}{1-\gamma } + b)^{\gamma } \\
&= \frac{1+\infty}{\infty } [\frac{ax}{1-\gamma }]^{-\infty} \\
&= \frac{\infty}{\infty } [\frac{ax}{1-\gamma }]^{-\infty} \\ \\
&= [\frac{(ax)^{\gamma }}{(1-\gamma )^{\gamma }} \\
ln[U(x)] &= \gamma \cdot ln[ax] - \gamma\cdot ln[1-\gamma ]\\
&= \gamma ( ln[ax] - ln[1-y] ) \\
&=- e^{-ax}
\end{align*}

\textbf{(d) Power}: $let b \rightarrow 0$: 
\begin{align*}
U(x) &= \frac{1-\gamma }{\gamma } (\frac{ax}{1-\gamma } + b)^{\gamma } = \frac{1-\gamma }{\gamma } (\frac{ax}{1-\gamma })^{\gamma } \\
&= \frac{1-\gamma }{\gamma } (\frac{a^{\gamma }x^{\gamma }}{(1-\gamma )^{\gamma }})  \\
\text{Let }& c= \frac{1-\gamma }{\gamma } (\frac{a^{\gamma }}{(1-\gamma )^{\gamma }} \implies U(x) = cx^{\gamma }
\end{align*}
\textbf{(e) Logarithmic} Let $a=1, b \rightarrow 0, \gamma \rightarrow 0$
\begin{align*}
U(x) &= \frac{1-\gamma }{\gamma } (\frac{ax}{1-\gamma } + b)^{\gamma }=\frac{1-\gamma }{\gamma } (\frac{x}{1-\gamma })^{\gamma } \\
&= \frac{1-\gamma }{\gamma } (\frac{x^{\gamma }}{(1-\gamma )^{\gamma }}) \\
lim_{\gamma\rightarrow 0} &\frac{1-\gamma }{\gamma } (\frac{x^{\gamma }}{(1-\gamma )^{\gamma }}) = lim_{\gamma\rightarrow 0} \frac{1-\gamma }{\gamma } \cdot  lim_{\gamma\rightarrow 0} (\frac{x^{\gamma }}{(1-\gamma )^{\gamma }})\\
\text{By Theorem 9.4 (Ross pg 47)} & \; \; \; \; \; lim(s_nt_n) = (lims_n)(limt_n)\\
\text{First:  }& lim_{\gamma\rightarrow 0} \frac{1-\gamma }{\gamma } = \frac{1}{\gamma } \\
\text{Second:  } & lim_{\gamma\rightarrow 0} (\frac{x^{\gamma }}{(1-\gamma )^{\gamma }})= (\frac{x^{\gamma }}{(1 )^{\gamma }})= x^{\gamma } \\
U(x) &= \frac{1}{\gamma }  x^{\gamma } \\
\frac{d}{dx}U(x) &= x^{\gamma - 1} \implies  lim_{\gamma\rightarrow 0} \frac{d}{dx}U(x) = \frac{1}{x}\\ 
\int \frac{d}{dx}U(x) =& U(x) = \int \frac{1}{x} = ln[x] \\
U(x)& = ln[x]
\end{align*}

\textbf{Risk Aversion Coefficient} 
%%%%%%%%%%% NOT DONE %%%%%%%%%%%%%%%%%%%%%%%%%

\end{problem}

\begin{problem}{9 Quadratic mean variance}. An investor with unit wealth maximizes the expected value of the utility function $U(x) = ax - bx^2/2 $ and obtains a mean-variance efficient portolio. A friend of his with wealth W and the same utility function does the same calculation but gets a different portfolio return. However, changing $b$ to $b'$ does the trick. \\
\textbf{Solution}
\begin{multicols}{2}
 Well investor 1 will maximize  
 
 $U(x) = ax - bx^2/2 $ \textbf{(1a)}
 
   subject to $P_x\cdot x =1$. \textbf{(2a)}
   
To maximize, take the derivative and set to zero:

$\ddx  ax - bx^2/2 = a-bx=0$

  So $x=a/b$  \textbf{(3a)}

  \columnbreak
 
     Well investor 2 will maximize  
 
 $U(x) = ax' - b'x'^2/2 $ \textbf{(1a)}
 
   subject to $P_{x'}\cdot x' =W$. \textbf{(2b)}
   
To maximize, take the derivative and set to zero:

$\ddx  ax' - b'x'^2/2 = a-bx'=0$

  So $x'=a/b'$ \textbf{(3b)}
\end{multicols}
Now solve (2b) for $P_x$ to get $P_x=W/x'$. Plug into (2a) to get $w/x'\cdot x = 1$ Then plug in (3a) to get $w/x' \cdot a/b  \implies b/w=a/x'$ Finally, plug in (3b) to yield: $b/w = a/(a/b') \implies b'=b/w$
\end{problem}

\end{document}



% Set the overall layout of the tree




\tikzstyle{level 1}=[level distance=3.5cm, sibling distance=3.5cm]
\tikzstyle{level 2}=[level distance=3.5cm, sibling distance=2cm]

% Define styles for bags and leafs
\tikzstyle{bag} = [text width=4em, text centered]
\tikzstyle{end} = [circle, minimum width=3pt,fill, inner sep=0pt]

\begin{tikzpicture}[grow=right, sloped]
\node[bag] {Bag 1 $4W, 3B$}
    child {
        node[bag] {Bag 2 $4W, 5B$}        
            child {
                node[end, label=right:
                    {$P(W_1\cap W_2)=\frac{4}{7}\cdot\frac{4}{9}$}] {}
                edge from parent
                node[above] {$W$}
                node[below]  {$\frac{4}{9}$}
            }
            child {
                node[end, label=right:
                    {$P(W_1\cap B_2)=\frac{4}{7}\cdot\frac{5}{9}$}] {}
                edge from parent
                node[above] {$B$}
                node[below]  {$\frac{5}{9}$}
            }
            edge from parent 
            node[above] {$W$}
            node[below]  {$\frac{4}{7}$}
    }
    child {
        node[bag] {Bag 2 $3W, 6B$}        
        child {
                node[end, label=right:
                    {$P(B_1\cap W_2)=\frac{3}{7}\cdot\frac{3}{9}$}] {}
                edge from parent
                node[above] {$B$}
                node[below]  {$\frac{3}{9}$}
            }
            child {
                node[end, label=right:
                    {$P(B_1\cap B_2)=\frac{3}{7}\cdot\frac{6}{9}$}] {}
                edge from parent
                node[above] {$W$}
                node[below]  {$\frac{6}{9}$}
            }
        edge from parent         
            node[above] {$B$}
            node[below]  {$\frac{3}{7}$}
    };
\end{tikzpicture}


\section{Definitions}
\underline{Def: Forward Rate Formulas} (pg 79). The implied forward rate between times $t_1$ and $t_2$ is the rate of interset between those times that is consistent with a given spot rate curve. For Yearly compounding, the forward rate is:  
\begin{align*}
f_{i,j} =& [\frac{(1+s_j)^j}{(1+s_i)^i}]^{1/(j-i)}-1 \\
 e^{s(t_2)t_2} =& e^{s(t_1)t_1}e^{f_{t_1,t_2}(t_2-t_1)}
\end{align*}

\underline{Discount Factor Relation} The discount facot between periods i and j is defined as $$ d_{i,j}=[\frac{1}{1+f_{i,j}}]^{j-i}$$ These factors satisfy the compounding rule: $d_{i,k}=d_{i,j}d_{j,k}$\\

\underline{Def. Derivative (Ross pg 223)} Let F be a real valued function defined on an open interval contained a point a. We say f is differentiable at a, or f has derivative at a if the limit $$ f'(a) = \lim_{x \to a} \frac{f(x)-f(a)}{x-a} $$




https://www.investopedia.com/university/advancedbond/bond-pricing.asp
https://quant.stackexchange.com/questions/22288/duration-of-perpetual-bond
http://people.stern.nyu.edu/gyang/foundations/sample-final-solutions.html
http://pages.stern.nyu.edu/~jcarpen0/courses/b403333/07convexh.pdf
https://web.stanford.edu/class/msande247s/2009/summer%2009%20week%205/Bond%20Formula%20Sheet.pdf


\underline{Def: Forward Rate Formulas} (pg 79). The implied forward rate between times $t_1$ and $t_2$ is the rate of interset between those times that is consistent with a given spot rate curve. For Yearly compounding, the forward rate is:  
\begin{align*}
f_{i,j} =& [\frac{(1+s_j)^j}{(1+s_i)^i}]^{1/(j-i)}-1 \\
 e^{s(t_2)t_2} =& e^{s(t_1)t_1}e^{f_{t_1,t_2}(t_2-t_1)}
\end{align*}

\underline{Discount Factor Relation} The discount facot between periods i and j is defined as $$ d_{i,j}=[\frac{1}{1+f_{i,j}}]^{j-i}$$ These factors satisfy the compounding rule: $d_{i,k}=d_{i,j}d_{j,k}$\\

\underline{Def. Derivative (Ross pg 223)} Let F be a real valued function defined on an open interval contained a point a. We say f is differentiable at a, or f has derivative at a if the limit $$ f'(a) = \lim_{x \to a} \frac{f(x)-f(a)}{x-a} $$



\begin{align*}
\text{Maximize  } & 4x_1 +5x_2 +3x_3 +4.3x_4 + x_5 + 1.5x_6 + 2.5x_7 + 0.3x_8 + x_9 + 2x_{10} \\
\text{Subject to } & 2x_1 + 3x_2 + 1.5x_3 + 2.2x_4 +0.5x_5 +15x_6 + 2.5x_7 +0.1x_8 + 0.6x_9 + x_{10} \leq 5 \\ 
& x_1 + x_2 + x_3 + x_4 \leq 1 \\
& x_5 + x_6 + x_7 \leq 1 \\
& x_8 + x_9 + x_{10} \leq 1 \\
\end{align*}